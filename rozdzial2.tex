\chapter{Przykłady rozwiązań}
\label{cha:przyklady_rozw}

\section{Monitorowanie pacjenta}
\label{sec:monitorowanie_pacj}
\subsection{Monitorowanie postury \cite{Lee:2012:MPM:2370216.2370320}}
Zaproponowany mechanizm szacuje różne wartości reprezentujące posturę użytkownika takie jak: kąt nachylenia szyi , odległość oglądania, stan wzroku użytkownika poprzez analizę danych z przedniej kamery , akcelerometru, czujnika orientacji lub dowolną ich kombinację. Powiadamia użytkownika jeśli oszacowane
wartości są utrzymywane w nieprawidłowym zakresie ponad dozwolony czas.

\subsection{Ocena zdrowia i zdolności kierowcy \cite{6583819}}
Aplikacja sprzężona z Data Loggerem zainstalowanym w pojeździe, który gromadzi dane z czujników znajdujących się w pojeździe i na ciele kierowcy. Aplikacja jest przeznaczona do monitorowania jakości sygnałów real-time i znacznego skrócenia czasu oszacowania stanu zdrowia kierowcy łącznie z jego zdolnościami do prowadzenia pojazdów. System składa się z dwóch modułów:
\begin{itemize}
\item smartbio 1 - system badania\\
Służy do ułatwienia gromadzenia informacji na temat rożnych badań fizjologicznych/psychologicznych i wykorzystanie ich do obliczenia współczynnika, określającego zdolność kierowcy do jazdy

\item smartbio 2 - system monitorowania\\
Dzieli się na dwie warstwy hardware i monitoring. Warstwa hardware gromadzi i wstępnie przetwarza dane (z sensorów). Warstwa monitoring ma postać aplikacji na smartfona, jej głównym zadaniem jest komunikacja między użytkownikiem, a warstwą hardware
\end{itemize}

\subsection{Diagnozowanie depresji \cite{5291726}}
 
Ta praca proponuje techniki poprawiające komunikacje w BSN (Body Sensor Network), które gromadzą dane o stanach emocjonalnych pacjenta. BSN  mogą stale monitorować, dyskretnie szacować i klasyfikować stany depresyjne. Dodatkowo dane na temat życia pacjenta mogą zostać skorelowane z uwarunkowaniami fizjologicznymi, aby zidentyfikować w jaki sposób poszczególne bodźce wywołują objawy. Taki ciągły strumień danych jest poprawą w stosunku do migawki objawów, które zaobserwuje lekarz w ciągu badania. 
 

%---------------------------------------------------------------------------

\section{Zarządzanie informacją}
\label{sec:zarzadz_inf}

\subsection{Zarządzanie chorobami przewlekłymi}
Jest to pewna znaleziona koncepcja aplikacji mobilnej. Celem pracy było zidentyfikowanie funkcji i wymagań funkcyjnych , które pomogły by użytkownikowi w zarządzaniu opieką nad swoją chorobą przewlekłą
 Projekt przewiduje kompleksowe działania takie jak wyświetlanie i zarządzanie lekami, harmonogram badań/wizyt, notatki, plany diet, ważne informacje o planie leczenia

\subsection{Zarządzanie lekami}
 Aplikacja do zarządzania podawaniem leków. To pamiętnik śledzący i zarządzający lekami w celu zapobiegania błędom medycznym. 
 Poprzez wizje,dźwięk,wibracje  SapoMed przypomina użytkownikom o ich harmonogramie leków.
 Aplikacja umożliwia rejestrowanie leków poprzez kamerę za pomocą barcode na opakowaniu. 
 Korzysta z usług internetowych (web services) aby uzyskać informacje o lekach, a nawet o ich dawkowaniu
 Wykorzystuje też web services do zapamiętywania spożywanych wcześniej leków.

\section{Problem bezpieczeństwa}
\label{sec:problem_bezp}