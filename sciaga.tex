\chapter{Ściąga z artykułami}
\label{cha:sciaga}

\cite{JoMIRPalmert2012DesignofanmHealthAppfortheSelf-managementofAdolescentType1Diabetes:APilotStudy} Artykuł traktuje o zastosowaniu aplikacji m-health przy schorzeniach cukrzycy typu 1 (nieinsulinozależnej). Ponieważ dla cukrzyków typu 2 aplikacje te wykazały bardzo dużą poprawę stanu zdrowia i świadomości użytkowników, postanowiono zbadać, jak sytuacja przedstawia się dla osób chorujących na cukrzycę typu 2.
\cite{IENEleanorS.Freshwater2014Technologyfortrauma:testingthevalidityofasmartphoneappforpre-hospitalclinicians} W tym artykule omówiona jest skuteczność aplikacji na smartphone'a w kwestii przydzielania pacjentów po urazie do odpowiednich placówek hospitalizacyjnych. Porównana jest ona z tradycyjną metodą, wymyśloną na potrzeby „sieci urazowej”, stworzonej w Anglii.
\cite{AjopDavidKimhy2014Useofmobileassessmenttechnologiesininpatientpsychiatricsettings} Przeprowadzono test użyteczności aplikacji m-health dla pacjentów cierpiących na poważne zaburzenia psychiczne – w tym wypadku schizofrenii. Badano wartość informacji uzyskanych dzięki takiej aplikacji w kontekście wartości medycznej.
\cite{JomIrBierbrier2014EvaluationoftheAccuracyofSmartphoneMedicalCalculationApps} Ponieważ dokładność aplikacji pracujących na danych medycznych nie była nigdy obliczana, postanowiono ją zbadać. Przetestowano ogólnodostępne aplikacje z takich serwisów jak Google Play, BlackBerry World i iTunes App Store.
\cite{MPiEKer-Cheng2014ASmartphoneAPPforHealthandTourismPromotionSmartphones;Tourism;Colleges&universities;Marketing;Historicbuildings&sites;Festivals;Parks&recreationareas;Onlineinstruction;Folklore;Scienceeducation;Ruralareas;Mountainclimbing} Tematem pracy jest aplikacja, której celami są edukacja , promocja turystyki oraz promocja zdrowia poprzez monitorowanie zużycia energii użytkownika.
\cite{AICPSSanders2013RemotesmartphonemonitoringformanagementofParkinsonsDiseaseEEGaccelerationactivitymonitoringelectroencephalogrammobilecomputingsensorsmart-phonewatchwrist} Monitorowanie choroby Parkinsona na podstawie badań prowadzonych w domu przy pomocy telefonu komórkowego typu smartphone oraz wykorzystywanie zebranych danych do efektywniejszej terapii.
\cite{POChaiyachati2013APilotStudyofanmHealthApplicationforHealthcareWorkers:PoorUptakeDespiteHighReportedAcceptabilityataRuralSouthAfricanCommunity-BasedMDR-TBTreatmentProgramHospitals;Tuberculosis;Studies;Intervention;Workers;Patients} Praca mówi o aplikacji do monitorowania pacjentów z gruźlicą oporną na wiele leków. Jej celem jest ocena działania aplikacji
\cite{BCDPfaeffli2012AmHealthcardiacrehabilitationexerciseintervention:findingsfromcontentdevelopmentstudiesCardiacrehabilitation;Exercise;Telemedicine;Internet}Stworzenie aplikacji mobilnej wspomagającej program „rehabilitacji serca” (cardiac rehabilitation) i badanie jej efektywności wśród pacjentów objętych tym programem.
\cite{BIRajan2012ThePromiseofWireless:AnOverviewofaDevice-To-CloudmHealthSolutionControlInstrumentationSurgicalimplantsMonitoringBusinessJourneysHealthcareSubsidiaries} Artykuł traktujący o tworzeniu i użytkowaniu aplikacji m-health, a także problemach i korzyściach wynikających z zastosowania technologii chmury.
\cite{ICHoneyman2014Mobilehealthapplicationsincardiaccare} Przegląd zastosowań aplikacji m-health dla osób cierpiących na przewlekłe choroby serca, takie jak np. arytmia. Zwrócenie uwagi na pomoc aplikacji w zwiększeniu świadomości w dziedzinie pierwszej pomocy przy atakach serca.
\cite{BISikka2012TheFutureofMedicine:TheEmergencyRoomAndMobileHealthInstrumentationTrackingMonitoringEmergenciesHealthcareHealthEmergencymedicalservicesHospitals} Wywiad ze specjalistami w dziedzinie m-health, wskazującymi na korzyści wynikające z globalnego zastosowania medycznych aplikacji mobilnych, takich jak np. wcześniejsza diagnoza w przypadku ataku osoby chorej na serce czy lepszy styl życia u osób chorych na cukrzycę, pozwalający zdecydowanie zmniejszyć terapię farmakologiczną.
\cite{MEPeck2012LivelifeuntetheredwithmobilehealthappsUnitedStates--USPhysiciansSmartphonesCellularPhonePhysiciansPrimaryCareSoftwareutilitiesHumansUnitedStatesElectronichealthrecordsPracticeManagementMedicalPortablecomputersMedicalInformaticsSoftware} Przedstawiony jest tu punkt widzenia lekarza korzystającego z aplikacji. Omówione są wszystkie udogodnienia, jakie oferuje korzystanie z telefonu w codziennej pracy.
\cite{PPRModi2013MobileHealthTechnologyinDevelopingCountries:TheCaseofTanzaniaWorldHealthOrganizationTanzaniaPublichealthHumanimmunodeficiencyvirus--HIVAcquiredimmunedeficiencysyndrome--AIDSCellulartelephonesDevelopingcountries--LDCs} Badanie wpływu użytkowania aplikacji m-health we wschodnioafrykańskiej Tazanii, gdzie istnieje duże ryzyko zarażenia rozmaitymi chorobami, także przewlekłymi. Teza, iż poprawa stanu zdrowia w kraju pozwoli na prężniejszy rozwój ekonomiczny.
\cite{PMFree2012TheEffectivenessofMobile-HealthTechnology-BasedHealthBehaviourChangeorDiseaseManagementInterventionsforHealthCareConsumers} Analiza korzyści wynikających z użytkowania medycznych aplikacji mobilnych w codziennym życiu – dane statystyczne, zgromadzone z wielu źródeł
\cite{PDaTMurad2014TheimpactofmobilehealthapplicationsonemergencymedicalservicesandpatientinformationprivacyHealthcaremanagementGeographicinformationscienceComputerscienceInformationTechnology} Autor zwraca uwagę na fakt, iż w większości aplikacji zdrowotnych sprawa prywatności i bezpieczeństwa danych użytkowników traktowana jest drugorzędnie, jako dodatek do aplikacji. Badana jest tu także metoda jak najefektywniejszej wymiany danych.
\cite{Bi&t/AftAoMILogan2012ARoundtableDiscussion:EmbracingtheMobileRevolutionIndexMedicus} Rozmowa z ekspertami w dziedzinie zdrowia i technologii bezprzewodowych. Poruszono takie kwestie jak odpowiednie urządzenia, połączenie aplikacji bezpośrednio ze szpitalem oraz przyczynę popularyzowania się technologii mHealth
\cite{NEPSkiba2014TheConnectedAge:MobileAppsandConsumerEngagementSmartphonesInternetaccessFDAapprovalDiseasecontrolElectronichealthrecordsHumansInformationmanagementStudentsNursingConsumerParticipationPatientsTextmessagingSocialnetworksWomenshealthHospitalsPersonalhealthGlobalpositioningsystems--GPS} Próba skierowania zainteresowania użytkowaniem smartphone'ów studentów (i nie tylko) w kierunku technologii wspomagających zachowania sprzyjające utrzymaniu dobrego stanu zdrowia.
\cite{PDaTWerkmeister2013TheUseofApplicationsonMobileDevicesinaMidwesternPopulationClinicalpsychologyPsychology} Analiza popularności użytkowania aplikacji mobilnych wśród osób cierpiących na różne problemy zdrowotne – jak choćby depresja czy dysfunkcje wynikające z nadużywania alkoholu
\cite{BIFacchinetti2012ThisProcessIsJustBeginning:ConnectingMobileMedicalDevicesTelemedicineWirelessTechnologyTechnologicalchangeDeliveryofHealthCareComputerCommunicationNetworksSystemsIntegrationMedicalequipmentInformationtechnologySoftware} Nieustannie postępujący rozwój technologii zmusza autorów artykułu do przeanalizowania przyszłości urządzeń medycznych pod kątem ich całkowitego połączenia, także w dziedzinie wymiany informacji, w przeciwieństwie do obecnej architektury bazującej na indywidualnych topologiach maszyn.
\cite{Ismaeel2013EffectiveSystemforPregnantWomenusingMobileGIS} Ze względu na dużą śmiertelność wśród ciężarnych kobiet w krajach, gdzie opieka medyczna nie jest ogólnodostępna, takich jak Afryka czy Azja, postanowiono zaproponować rozwiązanie problemu bazujące na aplikacji mHealth.
\cite{CDRJ.GrahamThomas2014ReviewofInnovationsinDigitalHealthTechnologytoPromoteWeightControlScience&TechnologyLifeSciences&BiomedicineEndocrinology&MetabolismENDOCRINOLOGY&METABOLISMRANDOMIZED-CONTROLLED-TRIALPHYSICAL-ACTIVITYLOSSPROGRAMLOSSMAINTENANCEMOBILE-TECHNOLOGYOBESITYEPIDEMICECONOMICBURDENUNITED-STATESPRIMARY-CAREUSADULTS} Postęp technologiczny w ostatnich latach spowodował drastyczny wzrost problemów zdrowotnych społeczeństwa wynikających z otyłości. Postanowiono wykorzystać ten rozwój na korzyść większej kontroli wagi wśród osób używających smartphone'ów na co dzień.
\cite{FMottl2014mHealthappoffersround-the-clockmedicalconciergesubscriptionservice} Opis nowo udostępnionej aplikacji, pozwalającej np. terminowo przepisywać pacjentowi odpowiednie leki, bazując na dotychczasowej historii choroby
\cite{SSahoo2012EfficientSecurityMechanismsformHealthApplicationsUsingWirelessBodySensorNetworks} Zaproponowanie architektury zapewniającej bezpieczeństwo transmisji danych medycznych z nowoczesnych przenośnych urządzeń monitorujących
\cite{FMottl2014Smartphoneappprovesvaluableforcardiacpatients} Zauważono, iż użytkowanie aplikacji mHealth przez pacjentów po poważnych urazach serca, takich jak zawał, zmniejsza ryzyko konieczności ponownego pojawienia się w szpitalu w ciągu 90 dni od wypisu.
\cite{FSlabodkin2013HomelesspatientsmaybenefitfrommHealth} Prowadzone badania wykazały, iż znaczna część bezdomnych posiada telefon komórkowy przynajmniej z funkcją odbierania i wysyłania wiadomości tekstowych oraz wykonywania połączeń. To pozwoliło wysnuć hipotezę, iż aplikacje mHealth znacznie pomogą w utrzymaniu stanu zdrowia takich osób na dobrym poziomie
\cite{FBartley2014mHealthapphelpselderlypatientswithmedicalindependenceAndroidiOSmedicaldevicesmHealthMobileapplicationsTablets} Osoby starsze często mają kłopoty z pamięcią, dlatego też regularne przyjmowanie przez nie leków staje się niekiedy dużym problemem. Zamiast więc angażować w leczenie osoby trzecie, autorzy artykułu proponują skorzystanie z dobrodziejstw urządzeń mobilnych.
\cite{FSlabodkin2013Kvedar:MobilemoodtrackersarepromisingmHealthtrend} Zwrócenie uwagi na nowy trend w dziedzinie mHealth, którym jest badanie nastroju pacjenta – obok standardowych pomiarów, takich jak ciśnienie krwi, puls, masa ciała, poziom cukru we krwi. Ma on pomóc w umożliwieniu szerszego spojrzenia na problemy zdrowotne pacjenta
\cite{BMacKellar2014AussiemhealthfirmtorollouttoolforParkinsons} Notka o technologii wspomagającej monitorowanie choroby Parkisona dzięki domowemu sprzętowi do monitorowania
\cite{BPHPetrella2014Mobilehealthexerciseandmetabolicrisk:arandomizedcontrolledtrialMetabolicdisordersOlderpeopleMedicalresearchCholesterol} Badanie sprawdzające, czy aplikacja mHealth jest w stanie zmniejszyć ryzyko problemów wynikających z nieprawidłowej przemiany metabolitycznej, takich jak choroby serca czy cukrzyca
\cite{IJoCCaSSWang2014TelemedicineBasedonMobileDevicesandMobileCloudComputing} Praca mówi o funkcjach smartfonów i tabletów oraz kwestii bezpieczeństwa ich informacji, aplikacjach i wyzwaniach w telemedycynie, telemedycynie bazującej na Mobile Cloud Computing i o jej wyzwaniach
\cite{ISoMAHN&CVarshney2014Mobilehealth:medicationabuseandaddictionaddictionmobilehealthmonitoringprescriptionabuse} Złe stosowanie przepisanych leków w bardzo wielu przypadkach prowadzi do uzależnień, co zwiększa koszty leczenia i pogarsza sytuację zdrowotną pacjentów. Ciągłe monitorowanie stanu zdrowia i dawkowania leków przy pomocy aplikacji mobilnych może prowadzić do rozwiązania tego problemu.
\cite{CoIaKMEklund2013Onchallengeswithmobilee-health:lessonsfromagame-theoreticperspectivegametheoryhealthinformaticsinformationretrieval} Ponieważ często szukanie informacji na temat choroby na podstawie symptomów okazuje się bardzo problematyczne – ogromna ilość wyników, niezbyt dokładna obserwacja symptomów ze strony samego użytkownika – postanowiono zastosować nowe rozwiązanie wyszukiwania bazujące na teorii gier.
\cite{HdmWetzel2012MobilehealthisintheregulatorycrosshairsHealthadministrationComputersHandheldUnitedStatesFoodandDrugAdministrationDeliveryofHealthCareUnitedStatesInformationStorageandRetrievalGovernmentRegulation} Wraz ze wzrostem dostępności oraz liczby aplikacji mHealth w ostatnich latach pojawiają się pytania o regulacje prawne dotyczące tej technologii. Omówiona jest klasyfikacja aplikacji pod kontem ryzyka które niesie ona względem pacjenta.
\cite{DTPeyton2013MobileappsformanaginghealthUnitedStates--USPatientcareplanningTechnologyadoptionPharmacistsSoftwareutilities} Przegląd aplikacji medycznych dostępnych na rynku zarówno dla pacjentów, jak i farmaceutów i lekarzy
\cite{OHTyler2014MobilehealthmonitorsUnitedKingdom--UKSmartphonesTelemedicineQualityofcareOccupationalhealth} Wzmianka o tym, jak działa i na czym polega „mobilne zdrowie”, omówienie konsekwencji użytkowania i kosztów, jakie technologia za sobą niesie
\cite{CoHFiCSPeyton2013MyMobileHealthMyMobileLife:methodsfordesigninghealthinterventionswithadolescentsHuman-centeredcomputingactionresearchadolescents&youthmobilehealthparticipatorydesignvulnerablepopulations} Przejście w systemie opieki medycznej z pediatrycznej na skierowany dla osób dorosłych dla pacjentów przewlekle chorych często jest kosztowne i prowadzi do powstawania braków w dokumentacji choroby. Tradycyjne próby jej poprawnego przetransferowania są bardzo kosztowne, dlatego też prowadzone są prace nad systemem mobilnym umożliwiającym „bezbolesne dojrzewanie medyczne”.
\cite{JoPNHSElias2014MobileAppsforPsychiatricNursesPsychiatric-mentalhealthnursingSmartphonesHandheldcomputersSoftware} Odpowiedzi na najczęściej pojawiające się pytania o aplikacje oferujące mobilne monitorowanie zdrowia w dziedzinie zdrowia psychicznego
\cite{OHPhillips2014Appsforhealthprofessionals} Opis niektórych ogólnodostępnych aplikacji na systemy iOS i Android dla mHealth